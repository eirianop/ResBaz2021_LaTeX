\documentclass[document.tex]{subfiles}
\begin{document}

\chapter{2021-11-26}

\subsubsection{Description of some Dissimilarity Measures}
Many measures incorporate relative abundance information, and the paper 
focusses on extensions to the two measures below:
\begin{itemize}
    \item Jaccard dissimilarity:
    \\$d_J = (b+c)/(a+b+c)$
    \\where 
    \begin{quote}
        $a$ is the number of species shared
        \\$b$ is the number of species in unit 1 that do not occur in unit 2
        \\$c$ is the number of species in unit 2 that do not occur in unit 1
    \end{quote}
    \item S{\o}rensen measure, also described by Dice:
    \\$d_S = (b+c)/(2a+b+c)$
    \\ this is monotonically related to the Jaccard measure (with $d_S < d_J$)
    \\ S{\o}rensen gives double value to shared species.
    \item both of these are favoured in ecology because they exclude joint 
    absences and can be interpreted in a probabilistic framework 
    (Chao \textit{et al.} 2005)
\end{itemize}

Extensions:
\begin{itemize}
    \item Bray \& Curtis (1957):
    \begin{itemize}
        \item $d_{BC} = \frac{\sum_{k=1}^{p}|\chi{}_{1k}-\chi{}_{2k}|}{\sum_{k=1}^{p}(\chi{}_{1k}-\chi{}_{2k})}$
        \item $\chi{}_{1k}$ is the abundance of species $k$ in sampling unit 1
        \item $\chi{}_{2k}$ is the abundance of species $k$ in sampling unit 2
        \item $p$ is the total number of species recorded across both units
        \item if the variables differ in their scales, then a transformation of abundances (e.g.\ square roots or fourth roots) are often applied first
    \end{itemize}

    \item Gower (1971, 1987)
    \begin{itemize}
        \item $d_{G} = \frac{\sum_{k=1}^{p}|\chi{}_{1k}-\chi{}_{2k}|/R_k}{\sum_{k=1}^{p}w_k}$
        \item $R_{k}$ is the range of the $k$th species 
        \item $w_k$ is an optional weight that can be given to each species
        \item dividing differences by the range (or std dev) is optional, 
        the idea is to remove differences in scale among variables.
    \end{itemize}

    \item Jaccard measure (from Chao \textit{et al.} 2005)
    \begin{itemize}
        \item interpretable as the probability of two species, one 
        drawn at random from each sample, will not be shared
        \item $d_{ChJ} = 1 - UV/(U + V - UV)$
        \item $U$ is the sum of the proportional abundances of species
        in sampling unit 1 that are shared with unit 2
        \item $V$ is the sum of the proportional abundances of 
        species in sampling unit 2 that are shared with unit 1
        \item $U = \sum_{k=1}^{p} w_k \frac{\chi{}_{1k}}{\chi{}_{1+}}$
        \item where $\chi{}_{1+}$ is the sum of abundances for all species
        in sampling unit 1 and $w_k=1$ if species $k$ occurs in both unit
        1 and unit 2 (shared), otherwise $w_k$ = 0
        \item $V$ is defined similarly, but for sampling unit 2.
        \item $d_{ChJ}$ is interpretable as the probability that two
        \textit{individuals} drawn at random from each sample, will not
        belong to a shared species. It takes values from 0 to 1 and 

        $d_{ChJ}=dJ$ for presence/absence data.
    \end{itemize}
\end{itemize}

\bib{}
\end{document}